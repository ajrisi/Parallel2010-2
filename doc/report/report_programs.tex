\section{Problem Solution}
This project involves two sets of deliverables, a sequential version utilizing Hirschberg's LCS algorithm and a parallel LCS algorithm provided developed by \cite{Liu:2007p402}.

\subsection{Sequential}
We investigated the single-processor, single-threaded version of an LCS algorithm using Hirschberg's algorithm\cite{Hirschberg:1975}. Refer to section \ref{sec:metrics} for benchmarks recorded for this sequential algorithm, compared against our parallelized algorithm. This code was written in Java using the Parallel Java libraries \cite{pj}.

Hirschberg's algorithm, developed by Dan Hirschberg in 1975, utilizes a dynamic programming model for finding the least cost sequence alignment between two string inputs. It is a divide and conquer version of the Needleman-Wunsch algorithm, and is used in bioinformatics for finding exact, maximal global alignments of DNA base pair sequences.\cite{wiki-hirsch}. The algorithm is considered superior to heuristic methods like BLAST or FASTA, but results in a space complexity of $O(min(n,m))$ space and time complexity of $O(mn)$ time, where $X$ and $Y$ are two input strings and $|x| = n$ and $|y| = m$.

\subsubsection{Algorithm Description}
Hirschberg's algorithm is divided into three distinct parts, Algorithm A, B and C. Algorithm B is an optimization of Algorithm A, thus our sequential version of the algorithm relies only on Algorithms B and C directly.

We introduce string $C = c_1c_2 \cdots c_p$ that is a subsequence of string $A$, where $A = a_1a_2 \cdots a_m$ if and only if there is a mapping $F{1, 2, \cdots, p} \rightarrow {1,2,\cdots,m}$ such that $f(i) = k$ when $c_i = a_k$ and F is a monotonically increasing function. Therefore, $C$ is a common subsequence of input strings $A = a_1a_2 \cdots a_m$ and $B = b_1b_2 \cdots a_n$ if $C$ is a subsequence of $A$ and $C$ is a subsequence of $B$\cite{Hirschberg:1975}. 

Algorithm $B$ produces an output vector $LL$, which is a derivation of Algorithm $A$ in \cite{Hirschberg:1975} such that the matrix $L$ as defined in that algorithm is reduced to requiring only rows $i$ and $i-1$, hence the use of a 2-row array $K$ of $j$-elements wide.

The space complexity of Algorithm $B$ as implemented in our Java version is $O(3n)$ deriving from the total allocated space of $3(n+1)$ integers for the three local storage arrays. The input arrays are pre-allocated and passed by reference. The time complexity of Algorithm $B$ is simply $O(mn)$ reflecting the nested for-loops executed $mn$ times.

\begin{codebox}
\Procname{$\proc{Algorithm-B}(m,n,A,B,LL)$}
\li	\Comment Initialization
\li $K[1][j] \gets 0 [j=0 \twodots n]$
\li \For $i \gets 1$ \To $m$ 
\li	\Do
\li 	$K[0][j] \gets K[1][j] (j=0\twodots n)$
\li 	\For $j \gets 1$ \To $n$ 
\li		\Do
\li			\If $A[i] = B[j]$ 
\li			\Then 
\li				$K[1][j] \gets K[0][j-1] + 1$
\li			\Else
\li				$K[1][j] \gets max(K[1][j-1], K[0][j])$
			\End
		\End
	\End
\li	$LL[j] \gets K[1][j] (j=0\twodots n)$
\end{codebox}

Algorithm $C$ recursively produces a solution to the LCS sub-problem, where at its finish returns the full longest common subsequence string. In our Java implementation of Algorithm $C$, the space complexity is proportional to $O(m + n)$ space. The time complexity of this algorithm is $O(mn)$, factoring in linear invocations of Algorithm-$B$ and -$C$ inside this function.

\begin{codebox}
\Procname{$\proc{Algorithm-C}(m,n,A,B,C)$}
\li \If $n = 0$ 
\li	\Then 
\li		\Comment If the problem is trivial, solve it
\li		$C \gets \epsilon$ \Comment Empty string
\li \ElseIf $m = 1$
\li	\Then
\li 	\For $j \gets 0$ \To $n$ 
\li     \Do
\li     	$C \gets A[1]$
\li		\End
\li \ElseNoIf
\li 	\Comment Otherwise, divide and conquer
\li 	$i \gets \lfloor m/2 \rfloor$
\li		\Comment Evaluate
\li		$\proc{Algorithm-B}(i, n, A_{1,i},  B_{1,n}, L1)$
\li		$\proc{Algorithm-B}(m-i, n, A_{n,i+1},  B_{n,1}, L2)$
\li		\Comment Find j
\li		\For $j \gets 0$ \To $n$
\li		\Do
\li			$M \gets \proc{max}(L1[j] + L2[n-j])$
\li			$k \gets \proc{min}(j)$ such that $L1(j) + L2(n-j) = M$
\li		\End
\li 	\Comment Solve simpler sub-problems
\li		$\proc{Algorithm-C}(i,k,A_{1,i}, B_{1,k}, C1)$
\li		$\proc{Algorithm-C}(m-i,n-k,A_{i+1,m}, B_{k+1,n}, C2)$
\li		\Comment Resulting output
\li		$C \gets \proc{strcat}(C1,C2)$
\li	\End
\end{codebox}

\subsection{Parallelized}
Research into the problem will help us determine the best LCS algorithm for parallelization. A modification of one of the algorithms chosen in the sequential portion of the project would be preferred for the most accurate analysis. The code will be written in C and parallelized using MPI (either OpenMPI or MPICH). Again, metrics described in section \ref{sec:metrics} will be used for our analysis of the chosen algorithm in its sequential and parallelized forms. It's not is not yet determined if we will investigate a pure MPI solution or consider a hybrid approach using MPI and PThreads.
