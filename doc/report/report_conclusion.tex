\section{Conclusion}
The results of the testing the dynamic sequential and the dynamic
solutions to the LCS length problem were very mixed.

The best example of the functionality of the parallel version can be
seen in the calculations for two 8192 byte strings. In the 8 kilobyte
test cases, the parallel version with one processor and the sequential
version operate in approximately equal time. The parallel version,
with additional processors, shows moderate speedup, increasing slowly
as the number of processors increases.

In the test cases of less than 8 kilobytes of data, the network
overhead of peer to peer communication lead to increasing computation
times as more processors were added. This caused the parallel version
to operate slower in almost all instances.

In the cases of more than 8 kilobytes of data, the network overhead
was no longer the limiting factor for time, rather, it was the actual
table calculation. As more data was used the comparisons, the
parallel version became extremely super-linear, and showed remarkable
speedup. This can be attributed to the JIT compiler effect. In the
calculation of the 128 kilobyte sequence similarities, the sequential
and single processor parallel versions were extrapolated, as the
Paranoia cluster only runs parallel software for an hour before
terminating execution. While the sequential version was unable to
calculate the similarity of two 128 kilobyte sequences within an hour,
the parallel version with 16 processors was able to calculate the
length of the longest common sequence in 3.65 minutes.
