\section{Paper Analysis: Boukerche et al }

\subsection{Problem Description}
This paper approaches the problem of finding an efficient means of
finding al longest common subsequences between two input
strings. While many algorithms use hueristics in order to estimate the
longest common subsequence (BLAST, FASTA), the algorithm described in
``An exact parallel algorithm to compare very long biological
sequences in clusters of workstations'' determines the exact length of
the longest common subsequences, as well as all instances of longest
common subsequences.

\subsection{Contributions}
``And exact parallel algorithm to compare very long biological
sequences in clusters of workstations'' provides a very concise
description of a parallel algorithm that calculates the longest common
subsequence instances in two phases - the first pass, that calculates
the coordinates in the similarity matrix of all longest common
subsequences, and the second pass that takes the list of all
coordinates and backtracks the longest common subsequence.

The first phase, the only one required to calculate the length of the
longest common subsequence, uses the \textit{wavefront} method of
parallelizing the calculation of the similarity table. By calculating
blocks of the table in parallel, the wavefront method calculates the
LCS length with minimal size and time requirements. The exact
algorithm used to calculate the LCS length in parallel is described
under the heading ``Dynamic-Programming-LCS-Length-Clu''.

Additionally, this paper provided the second phase. During the second
phase, a list of the end locations of the LCSs is gathered from each
processor. Once the global LCS length is calculated, coordinates are
distributed to workers who then use the backtracking algorithm used in
Hirschbergs algorithm to calculate teh actual LCS.

\subsection{Investigative Use}
The first pass of ``An exact parallel algorithm...'' describes the
\textit{wavefront} method, a means of parallelizing the calculation of
the similarity table, and additionally, the length of the longest
common subsequence. The algorithm described in this paper was used to
calculate the longest common subsequence length.
